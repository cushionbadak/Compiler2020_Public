\section {\SubC \ Testcases for Compiler Assignment}

이 섹션은 \SubC 언어를 CMachine 에서 사용할 수 있는 기계어로 바꾸는 과제 채점에 쓰일 \SubC 프로그램의 모습을 다룬다.

\begin{itemize}
  \item main 함수가 따로 없다.

  코드는 위에서 아래로 주욱 실행되는 것으로 여겨지며, 다 실행된 뒤에는 오류 없이 멈춰야 한다.
  
  \item 정수는 부호 있는 63-bit 정수형이며, overflow와 underflow가 일어나는 프로그램은 없다.
  
  \SubC 프로그램의 정수값은 OCaml 기본 정수형 구현을 따라간다.
  OCaml 기본 정수형은 63, 31 bit 부호있는 정수이며(시스템에 따라 다르다), 채점 환경은 63 bit 부호 있는 정수형을 사용한다.
  채점에 쓰이는 프로그램들은 31bit 정수값 계산에 대해 overflow, underflow가 일어나지 않을 것이다.

  \item 함수를 값처럼 사용하지 않는다.
  
  \SubC 문법은 함수의 타입을 표시할 수 있는 방법을 제공하지 않으며, 따라서 변수에 함수 포인터를 담을 수도 없다.
  시스템 콜로 미리 정의된 함수들과 프로그램에서 정의한 함수들은 \Expression 에서 인자를 넘겨주며 호출하는 방식으로만 사용한다.

  \item 함수를 정의할 때 항상 \mbtt{return} 명령문을 통해서 함수 내부 실행이 끝난다.
  \item 함수 정의할 때 정해진 인자 갯수와 타입을 지켜 함수를 호출한다.
  \item 배열 타입값을 인자로 받는 함수는 정의하지 않는다. 포인터나 구조체는 인자로 받을 수 있다.
  \item \CCif , \CCfor , \CCwhile , \CCswitch 문장의 조건문을 계산하면 항상 정수값이 나온다.
  \item \CCif 명령문에 대해 \CCbreak 나 \CCcontinue 명령문을 사용할 수 없다.
  \item \CCswitch 명령문에 대해 \CCbreak 명령문은 사용할 수 있지만 \CCcontinue 명령문은 사용할 수 없다.
  \item \CCswitch 명령문에 들어가는 case들은 그 순서나 연속성이 전혀 보장되지 않는다. case 문장의 조건 정수의 격차는 아주 크지는 않다.
  
  예를 들어, \CCswitch 명령문 안에 5, 1, (-1), 7 조건의 case 문장들과 default case 문장이 들어있을 수 있다.
  case 문장의 조건 정수의 격차가 아주 크지 않다는 뜻은 조건 정수의 최대 크기차가 1000을 넘지 않는다는 뜻이다.
  예를 들어, \CCswitch 명령문 안에 (-1000), 5, 1000, (-1) 조건의 case 문장들이 들어간 프로그램은 
  조건 정수의 최대 크기차가 2000 이므로 채점 프로그램으로 쓰이지 않는다.

  \item 대입문의 왼쪽 자리에 \CCrparen \CCampersand \LeftValueExpression \CClparen 는 사용하지 않는다.

\end{itemize}
