\section{\SubC \ Grammar Details}

이 섹션은 \SubC 의 문법과 의미를 상향식으로 자세히 설명할 것이다.


\subsection{Comment}

/* 와 */ 사이에 있는 문자열은 주석이다. /* 와 */ 사이에 대부분의 문자가 들어갈 수 있으며, / 문자는 허용되지 않는다.

\begin{lstlisting}[language=C, caption=Comment]
/* this is SubC comment example */
int x;
\end{lstlisting}




\subsection{Variable and Integer}

\Variable 은 영문자로 시작하는 문자열이다. 변수와 식별자를 표시할 때 쓰인다.
첫번째 글자는 영문자와 밑줄만 허용하지만, 두번째 글자부터는 숫자도 포함할 수 있다.

\medskip

\Integer 는 정수값을 표현할 때 사용한다. 십진수 표기법을 따르며, `0' 만 표현하지 않는 이상 맨 앞 글자는 0이 될 수 없다.
\SubC의 정수값은 유한한 범위의 부호 있는 정수이다.

\begin{lstlisting}[language=C, caption=Variable and integer]
int _x;
int[100] compilerGrade_;
int zero;

compilerGrade_[10] = 100;
_x = -10;
zero = 0;
\end{lstlisting}
  



\subsection{Type Expression}

\SubC 의 기본 타입은 \CCint \ 하나밖에 없으며, 이외에는 배열, 구조체, 포인터 타입을 쓸 수 있다.
\mbtt{typedef} 키워드는 지원되지 않지만, 대신 \mbtt{struct [name]} 의 사용자 정의 구조체 이름으로 새 타입이름을 만들 수 있다.

\begin{itemize}
  \item 배열
  
  다른 타입 표현 뒤에 대괄호와 숫자를 이용해서 배열 타입을 만들 수 있다. 
  예를 들어, \CCint \CClsbracket 10 \CCrsbracket 은 크기 10의 정수 배열 타입을 의미한다. 
  C의 배열 타입은 \CCint \ \mbtt{v} \CClsbracket 10 \CCrsbracket 처럼 변수 뒤에 대괄호가 위치하지만, 
  \SubC 의 배열 타입은 변수 앞에 대괄호가 위치한다.
  
  \item 포인터
  
  다른 타입 표현 뒤에 별표를 넣어 포인터 타입을 만들 수 있다.
  예를 들어, \CCint \CCasterisk 는 정수 포인터 타입이다.
  \SubC 의 포인터 타입 크기는 항상 1이다. 
  
  \item 구조체 및 사용자 정의 구조체 타입
  
  구조체 타입은 C의 구조체 문법과 유사하게 표현 가능하다.
  \CCstruct 키워드로 시작해서 이름이 있는 구조체 타입, 이름이 없는 구조체 타입 모두 표현 가능하며,
  한 번 등장한 이름이 있는 구조체 타입은 같은 이름 범위(scope) 안에서는 
  \mbtt{struct [name]} 과 같이 중괄호 부분을 생략해 적을 수 있다.
  
\end{itemize}


\begin{lstlisting}[language=C, caption=Type expression examples]
int x;        /* integer */
int[10] xs;   /* length-10 integer array */
int* p;       /* integer-pointer */

/* struct type with no name */
struct {
  int xx;
  int yy;
} point2d;

/* custom type struct-student array. The length of this type is 30.  */
struct student {
  int age;
  int grade;
  /* recursively defined struct type is allowed only if the size of the type is predictable. */
  struct student * friends;
}[10] students; 

/* the usage of defined struct type name */
struct student jisuk;
\end{lstlisting}




\subsection{Integer Operator}

\SubC 의 값 연산자들은 전부 정수 값을 다루는 연산자들이며, 정수 연산, 논리 연산, 비교 연산 세 종류로 이루어져 있다.
연산자의 의미는 C와 같다.
정수 연산에는 사칙연산과 나머지 연산이 있다. 
논리 연산에는 and, or, not 연산자가 있으며 0은 거짓, 0 아닌 값은 참으로 해석된다.
비교 연산에는 less-than, less-or-equal, greater-than, great-or-equal 연산자가 있다.
비트 연산자는 없다.


\begin{lstlisting}[language=C, caption=Operator examples]
int x; int y; int z;

x = 3; y = 5;

z = x + y;  /* z -> 8 */
z = y % x;  /* z -> 2 */
z = x && y; /* z -> true */
z = ! z;    /* z -> false (0) */
\end{lstlisting}




\subsection{Left Value}

문법에서 \LeftValueExpression 로 표현하는 Left Value는 대입문의 왼쪽에 등장할 수 있는 값이어서 Left Value라고 부른다.
미리 선언되었던 변수들이 대표적이며, 역참조, 참조, 배열 주소 접근, 구조체 이름 접근 방식으로 주소값을 만들 수 있다.
구문 해석의 모호함을 없애기 위해 역참조와 참조 연산자를 사용할 때에는 괄호로 감싸 사용하도록 설계되었다. 
주의할 점으로는 포인터 타입의 주소에 배열 주소 접근을 할 때는 암묵적으로 포인터 값을 역참조 한 뒤 배열 주소 접근을 한다는 점이 있다.
또한 일반적으로 Left Value가 대입문의 왼쪽이 아닌 값으로 쓰일 때에는 대부분 메모리의 해당 주소값이 가리키는 값을 내놓지만,
배열 타입의 Left Value는 주소값 자체를 내놓는 점을 주의해야 한다.


\begin{lstlisting}[language=C, caption=Left value examples 1]
int[3] a;              /* consider that the address of _a_ is 5 */
int *b;

/* consider that the address of _v1_ is 9  */
struct vec2 {int x; int y;} [3] v1;  
struct vec2 * v2;     /* consider that the address of _v2_ is 15 */

b = malloc(3);        /* consider that b points to 1021 */
b[0] = 30;            /* memory[1021] = 30      */
b[1] = 300;           /* memory[1022] = 300     */
b[2] = 3000;          /* memory[1023] = 3000    */

a[0] = b + 10000;     /* memory[5]    = 11021   */
a[1] = (*b) + 20000;  /* memory[6]    = 20030   */
a[2] = b[1] + 40000;  /* memory[7]    = 40300   */

v1[0].x = 4;          /* memory[9]    = 4       */
v1[0].y = 5;          /* memory[10]   = 5       */
v2 = v1;              /* memory[15]   = 9       */
v2[1].x = 10;         /* memory[11]   = 10      */
v2[2].x = 20;         /* memory[13]   = 20      */
\end{lstlisting}

\begin{lstlisting}[language=C, caption=Left value examples 2]
int a; int b;
a = 100;  b = 200;

int swap(int * x, int * y) {
  int tmp;
  tmp = (*x);
  (*x) = (*y);
  (*y) = tmp;

  return 0;
}

swap((&a), (&b));  
\end{lstlisting}




\subsection {Expression}

\SubC의 표현식은 계산 이후 값을 가지는 모든 표현을 의미한다.
일반적인 표현식처럼 정수값, 변수값(암묵적인 주소 참조), 연산자가 들어있는 식들을 포함한다.
또한 부작용이 있는 대입문이나 시스템 함수, 일반 함수들 또한 계산 이후 값을 가지므로 표현식으로 여겨진다.

\begin{lstlisting}[language=C, caption=Expression examples]
int a; int b; int c; int d;
a = 1; b = 2; c = 3; d = 4;

int f (int x, int y) {
  return x + y;
}

a = 3;        /* integer expression           */
b = a;        /* lvalue expression            */
c = 3 - 2;    /* binary operation expression  */
c = (! a);    /* unary operation expression   */
a = d = (3 + a);  /* multiple assigments in a line is allowed */
d = f(a, c);  /* function call expression     */
\end{lstlisting}




\subsection {System Call}

\SubC에는 다섯가지 시스템 콜 함수가 있다. 
\begin{itemize}
  \item \mbtt{malloc(n)} 은 n만큼(메모리 한 칸에는 하나의 정수값이 들어갈 수 있다) 힙 메모리를 할당하고 그 첫 주소를 반환한다.
  \item \mbtt{free(n)} 는 주소 n에 할당된 메모리를 회수하고 값 n을 반환한다.
  \item \mbtt{read()} 는 standard input 으로 정수값을 읽어 반환한다.
  \item \mbtt{write(n, 0)} 은 정수 n을 standard output 으로 출력한다.
  \item \mbtt{write(0, 1)} 은 whitespace character 하나를 standard output 으로 출력한다.
  \item \mbtt{write(0, 2)} 는 newline character 하나를 standard output 으로 출력한다.
  \item \mbtt{random()} 은 0 부터 1023 사이의 정수값 하나를 무작위로 반환한다.
\end{itemize}




\subsection {Statement}

\SubC의 명령문은 함수를 제외한 프로그램 실행 순서를 결정하는 표현들을 의미한다.
문법적으로는 모두 $\CCscolon$ 이나 $\CCrbrace$ 로 끝난다.
\CCbreak 는 가장 가깝게 명령문을 감싸고 있는 \CCswitch , \CCwhile , \CCfor 문을 벗어나고,
\CCcontinue 는 가장 가깝게 명령문을 감싸고 있는 \CCwhile, \CCfor 문에 다시 진입한다.
\CCswitch 문장은 default case 하나와 적어도 하나의 case 문장을 가지고 있어야 한다.
또한 \CCswitch 내부의 문장은 기본적으로 fallthrough 의미를 가진다.

\begin{lstlisting}[language=C, caption=Expression examples]
int i;
int a; int b; int c; int d;
a = 2; b = 4; c = 7; d = 10;
if(a == b) {
  b = 2;  /* the program does not reach here */
} else {
  for (i = 1; i < 7; i = i + 1)
  {
    b = b + i;
    if ((b % 3) == 0) { 
      break; /* the program does not reach here */ 
    }
  }
}
while(i > 0)
{
  b = b + (a * i);
  if(b < 100) {continue;}
  i = i - 1;
}

/* b will be 151 at this point */
switch (b)
{
  case 151:
    d = b;
    break;
  default:
    d = random();
}
\end{lstlisting}




\subsection {Function Definition}

\SubC의 함수 정의는 C와 거의 다르지 않다. 
타입 표현이 제한적이어서 void return type을 가질 수 없으며, 명령문 등장 전에 모든 지역 변수들이 선언되어야 한다.




\subsection {Program}

\SubC 프로그램은 세 요소로 구성된다. 전역 변수 선언, 함수 정의, 그리고 명령문이다. 
전역 변수는 항상 프로그램의 제일 앞에 모두 선언되어야만 한다. 
전역 변수 선언들 뒤에 올 함수 정의와 명령문들은 순서에 상관없이 먼저 쓰인 내용이 먼저 실행된다.


\begin{lstlisting}[language=C, caption=Quicksort function example]
int[8] a;
int    i;

for (i = 0; i < 8; i = i + 1) {
  a[i] = random();
  write(a[i], 0);
  write(0   , 1);    /* second_argument 1 whitespace */
}
write(0, 2);         /* second_argument 2 newline    */

int quicksort (int* v, int low, int high) {
  int pivot; int l; int h; int tmp;

  if (low >= high) { return 0; }

  if ((low + 1) == high) {
    if(v[low] > v[high]) {
      pivot   = v[low];
      v[low]  = v[high];
      v[high] = pivot;
    }
    return 0;
  }

  /* initialize */
  pivot = v[low];
  l     = low + 1;
  h     = high;

  /* loop */
  while (l < h) {
    /* move low high pointers */
    while ((v[l] < pivot) && (l < high)) { l = l + 1; }
    while ((v[h] >= pivot) && (h > low)) { h = h - 1; }
    if (l >= h) { break; }

    tmp  = v[l];
    v[l] = v[h];
    v[h] = tmp;
  }

  /* change pivot and v_h ___ since v_l will be greater than v_h */
  if (h > low) {
    v[low] = v[h];
    v[h]   = pivot;
    
    quicksort (v, low, (h - 1));
    if (h < high) { quicksort (v, (h + 1), high); }
  }
  else {
    quicksort (v, (low + 1), high);
  }

  return 0;
}

quicksort(a, 0, 7);

for (i = 0; i < 8; i = i + 1) {
  write(a[i], 0);
  write(0   , 1);    /* second_argument 1 whitespace */
}
write(0, 2);         /* second_argument 2 newline    */
\end{lstlisting}



